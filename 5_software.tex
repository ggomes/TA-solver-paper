\begin{figure}[h]
    \centering
    \includegraphics[width=\linewidth]{figs/Software_Block_Diagram.PNG}
    \caption{Software Framework Architecture}
    \label{fig:Block_Diagram}
\end{figure}

Using the unified VI DTA formulation and the generic iteration shown in \ref{fig:iteartion}, we designed a modular and extensible
software framework to solve DTA problems. As shown in Figure~\ref{fig:Block_Diagram}, the software framework has two main modules: the Model Manager and the Solver modules. The Demand Assignment and Path Costs components are wrappers for demand assignment $h$, and for path travel costs respectively $c$.

The Model Manager has three components that correspond function $F$: the Traffic Model, Cost Function and Sum components. The State Trajectory module is a wrapper for link states (e.g: flow per link), and the Link Cost module is a wrapper for the travel cost per link (e.g experienced travel time on a link). The Model Manager's role is to translate an assignment $h$, specified as a sequence of demand per path by the Demand Assignment module, into the corresponding path costs, represented as a sequence of cost per path.

The Solver module serves as an interface to plug in solution algorithms for DTA. The generic SOLVER works in a loop in which it generates candidate demand assignments, and expects to be given the corresponding network path costs. This loop continues until an equilibrium demand assignment is reached. 

The advantage of our modular software framework is that it can be easily extended to address different DTA problems. The Traffic Model and Cost Function components interfaces enable a user to plug in different traffic models and cost functions to address diverse DTA problems. For example, the software framework currently include three traffic models: ST, MN, and CTM, which all integrate in the framework via the Traffic Model component. The framework has a travel time based cost function. A user can add an energy or emission based cost function. The Solver interface allows to incorporate different DTA solution algorithm. We included three solver algorithms: the MSA, the FW, and the EPM, which can be applied depending on the $F$ function properties. Finally, we also designed the whole Model Manager component as an interface, enabling a user to integrate an implementation of the $F$ composition function without having to use the Traffic model, Cost Function and Sum modules. This is useful for problems, such as simulation-based DTA problems, where the traffic model is tightly coupled with the cost function evaluation. The complete documentation on software framework, with installations instructions can be found at ...

