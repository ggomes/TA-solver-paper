\documentclass[conference]{IEEEtran}
\IEEEoverridecommandlockouts
% The preceding line is only needed to identify funding in the first footnote. If that is unneeded, please comment it out.
\usepackage{cite}
\usepackage{amsmath,amssymb,amsfonts}
\usepackage{algorithmic}
\usepackage{graphicx}
\usepackage{textcomp}
\usepackage[utf8]{inputenc}
\usepackage{hyperref}
\usepackage{amsthm, amssymb,amsmath}
\usepackage{mathrsfs}
\usepackage{mathtools}
\usepackage{pgfplots}
\usepackage{todonotes}
\usepackage{dsfont}
\usepackage{enumitem}
\usepackage[capitalise,nameinlink]{cleveref}
\usetikzlibrary{shapes}


\renewcommand{\:}{\mathrel{\coloneqq}}
\renewcommand{\=}{\ensuremath{\eqqcolon}}
\renewcommand{\epsilon}{\varepsilon}
\usepackage[caption=false, font=normalsize, labelfont=sf, textfont=sf]{subfig}
\newcommand{\dd}{\ensuremath{\,\mathrm{d}}}
\newcommand{\ddt}{\ensuremath{\tfrac{\dd}{\dd t}}}

\newcommand{\R}{\ensuremath{\mathbb{R}}}
\newcommand{\N}{\ensuremath{\mathbb{N}}}

\newcommand{\rr}{\text{r}}
\newcommand{\nr}{\text{nr}}

\date{\today}
\newtheorem{Definition}{Definition}
\newtheorem{Routing}{Routing}
\newtheorem{Results}{Results}
\newtheorem{Lemma}{Lemma}
\newtheorem{Corollary}{Corollary}
\newtheorem{Assumption}{Assumption}

\newtheorem{Remark}{Remark}
\newtheorem{Example}{Example}
\newtheorem{Theorem}{Theorem}
\newtheorem{Challenge}{Challenge}
\newtheorem{Proposition}{Proposition}

\newcommand{\0}{\ensuremath{\boldsymbol{0}}}
\newcommand{\A}{\mathcal{A}}
\newcommand{\V}{\mathcal{V}}
\newcommand{\C}{\mathcal{C}}
\newcommand{\cS}{\mathcal{S}}
\newcommand{\T}{\mathcal{T}}

\newcommand{\fR}{\mathfrak{R}}
\newcommand{\fRD}{\mathfrak{RD}}
\newcommand{\fRL}{\mathfrak{RL}}
\newcommand{\fRU}{\mathfrak{RU}}


\newcommand{\source}{\mathcal O}
\newcommand{\sink}{\mathcal D}
\newcommand{\OD}{\source\sink}
\newcommand{\Aout}{\mathcal{A}_{\textnormal{out}}}
\newcommand{\Ain}{\mathcal{A}_{\textnormal{in}}}
\newcommand{\vout}{v_{\text{tail}}}
\newcommand{\vin}{v_{\text{head}}}
\newcommand{\paths}{\mathcal{P}}
\newcommand{\bx}{\boldsymbol{x}}
\newcommand{\bg}{\boldsymbol{g}}
\newcommand{\bu}{\boldsymbol{u}}
\newcommand{\btau}{\boldsymbol{\tau}}
\newcommand{\btheta}{\boldsymbol{\theta}}
\newcommand{\bb}{\boldsymbol{b}}
\newcommand{\bh}{\boldsymbol{h}}
\newcommand{\bk}{\boldsymbol{k}}
\newcommand{\bm}{\boldsymbol{m}}
\newcommand{\bp}{\boldsymbol{p}}
\newcommand{\bs}{\boldsymbol{s}}
\newcommand{\bX}{\boldsymbol{X}}
\newcommand{\app}{\text{r}}
\newcommand{\napp}{\text{nr}}
\newcommand{\shortp}{\mathcal{Y}}
\newcommand{\ord}{\boldsymbol{\text{Ord}}}
\newcommand{\e}{\mathrm{e}}
\DeclareMathOperator*{\argmin}{arg\,\min}
\newcommand{\ext}{\textbf{\textnormal{ext}}}


\newcommand{\Id}{\mathrm{Id}}
\newcommand{\bcX}{\boldsymbol{\mathcal{X}}}
\newcommand{\Int}{\ensuremath{\int\limits}}

\newcommand{\len}{\textnormal{len}}
\newcommand{\todoAll}[1]{\todo[inline,color=red!30!yellow]{All: #1}}
\begin{document}

\title{A software framework for solving dynamic traffic assignment\\
{\footnotesize \textsuperscript}
\thanks{Identify applicable funding agency here. If none, delete this.}
}

\author{\IEEEauthorblockN{First author given Name Surname}
\IEEEauthorblockA{\textit{dept. name of organization (of Aff.)} \\
\textit{name of organization (of Aff.)}\\
City, Country \\
email address}}

\maketitle

\begin{abstract}
\end{abstract}

\begin{IEEEkeywords}
\end{IEEEkeywords}

\section{Introduction}

\section{Variational Inequality for DTA}

\section{Software Framework Description}

\section{Traffic Models}
\subsection{Static Model}
\subsection{MN Model}
\subsection{Point Queue Model}

\section{Implemented Algorithms}

\subsection{Method of Successive Averages}
The Method of Successive Averages is a heuristic-based algorithm and does not put any restriction on $F$. However,it does not guarantee convergence to the optimal solution for the general VI problem \cite{nie2010solving}. The algorithm's steps are described below:
\begin{enumerate}
    \item Start with a feasible assignment $h^1$ and set $k=1$.
    \item Calculate the all-or-nothing assignment $y^k$.
    \item If ${\frac {\langle F(h^k),y^k-x^k \rangle} {\langle y^k, F(x^k)\rangle}} \leq
    \epsilon$, stop. Otherwise, go to step 4.
    \item set $\alpha = 1/k$, and $h^{k+1} = (1-\alpha)h^k + \alpha y^k$. 
    \item Set $k = k+1$ and go back to step 2.
\end{enumerate}

\subsection{Frank-Wolfe Algorithm}
The Frank-Wolfe Algorithm applies when $F$ is convex and continuously differentiable. It has the following steps:
\begin{enumerate}
\item Start with a feasible assignment $h^1$ and set $k=1$.
\item Determine the all-or-nothing assignment $y^k$ given the current assignment $h^k$.
\item Terminate if $\nabla F(h^k)(h^k-y^k) \leq \epsilon$ stop; otherwise, go to step 4.
\item Calculate $d^k = y^k - h^k$
\item Set $h^{k+1} = h^k +\alpha d^k$, where $\alpha$ is a solution the one-dimensional problem: $\min_{0 \leq \alpha \leq 1}G(h^k + \alpha d^k)$ and $G = \int_h F$.
\item Set $k = k+1$ and go back to step 2.
\end{enumerate}

\subsection{Extra Projection Method}
The Extra Projection Method (EPM) is based on the Euclidean Projection define as $\Pi_\mathcal{H} = argmin\{\lVert h-h^0\rVert:h \in\mathcal{H} \}$ and $\lVert.\rVert$ is the Euclidean norm. The EPM guarantees convergence when $F$ is Lipschitz continuous and pseudo monotone \cite{nie2010solving}. The EPM steps are described below:
\begin{enumerate}
\item Start with a feasible assignment $h^1$ and set $k=1$.
\item If ${\frac {\langle F(h^k),y^k-x^k \rangle} {\langle y^k, F(x^k)\rangle}} \leq \epsilon$, where $y^k$ is the all-or-nothing assignment, then stop. Otherwise, go to step 3.
\item Find $z^k = \Pi_\mathcal{H}(h^k - \tau F(h^k))$, , where $\tau < L$ and $L$ is the Lipschitz constant for $F$.
\item Find $h^{k+1} = \Pi_\mathcal{H}(h^k - \tau F(z^k))$, where $\tau$ is a above.
\item Set $k = k+1$ and go back to step 2.
\end{enumerate}

Since $L$ can be difficult to predetermine in practice, \cite{nie2010solving} propose the following algorithm to update $\tau$ after obtaining solution $h^{k+1}$:
\begin{enumerate}
    \item Evaluate $\phi(h^{k+1})$ and $\phi(h^k)$, where $\phi(h)$ is the all-or-nothing assignment given $h$.
    \item if $\phi(h^{k+1})-\phi(h^k) < 0$ and $\frac{|\phi(h^{k+1})-\phi(h^k)|}{|\phi(h^k)|}> \epsilon$, set $\tau = \tau \times \sigma$, where $\epsilon$ and $\sigma$ are positive scalars between 0 and 1.
\end{enumerate}

\section{Numerical Results}

\section{Conclusions}\label{sec:concl}

\section*{Acknowledgment}


\bibliographystyle{alpha}
\bibliography{citation.bib}

\end{document}
