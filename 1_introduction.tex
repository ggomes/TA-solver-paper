The term \textit{traffic assignment} (or route choice, or route assignment) captures a large array of problems concerning the distribution of traffic over a network. Traffic assignment problems arise in transportation planning applications, such as in the traditional ``four-step'' procedure, where it occupies the final step, following trip generation, trip distribution, and mode choice \cite{mcnally2007four}.  The fundamental principle that guides most traffic assignment models is Wardrop's principle \cite{wardrop1952some}, which states that amongst the available alternatives, drivers will select routes that minimize their travel times. Because the speed of traffic tends to decrease as the number of vehicles on the route increases, Wardrop's principle leads to an iterated game in which drivers test new routes, day after day, until they converge to a Nash equilibrium, a.k.a. a \textit{user equilibrium}. The term is also used for problems that involve a single decision maker, who prescribes the routes for users in order to optimize some system-level performance metric, e.g. total travel time, emissions, energy consumption. These are mathematical programs, and their solutions are known as \textit{system optimal} assignments. We will focus in this paper on user equilibrium traffic assignments.

Since the pioneering work of Wardrop, the mathematical and numerical study of traffic assignment has progressed at a moderate pace. This is perhaps mostly due to a historical scarcity of traffic data and network information, but also because of the difficulty of the problem. [CCC] gives several examples in which problems could not be solved for networks of useful size. It is not surprising therefore that the problem has received renewed interest with recent increases in traffic and network data, and computational power. Also driving this renewal are concerns about the `unintended consequences' of the widespread use of routing apps \cite{traffic_apps}, as well as advances in autonomous driving technologies.

An investigation into the techniques of traffic assignment reveals a multitude of models, numerical methods, and performance metrics. As is usually the case, the more detailed the model, the more expensive the computation. However it is not clear under which circumstances the additional effort (or investment in computational resources) is necessary. For example, although it is usually agreed that dynamical models that capture the essential phenomenon of congestion are desirable, it is also true that the computation (as well as network configuration effort) involved in solving dynamic traffic assignment (DTA) problems can be large. On the other hand, existing static planning methods cannot fully model traffic congestion because of their inability to represent queue formation and dissipation when traffic demand exceeds road capacities\cite{nie2010solving}. 

This paper presents a software framework for addressing such questions. The software is based on the formulation of traffic assignment as a \textit{variational inequality}, introduced by Nagurney \cite{nagurney2013network}. This is a very general formulation, as it requires only continuity (\XXX TRUE?) of the traffic model. Hence, it covers a wide range of models: static and dynamic, macroscopic and microscopic, 'analytical' and 'simulation-based', etc. The software is modular and extensible, in the sense that it provides interfaces for incorporating new models, cost functions, and numerical solvers. 

There are many other programs, both commercial and shared, that solve dynamic traffic assignment problems. 

\ggnote{I want to de-emphasize the analytical vs simulation-based dichotomy because I don't understand it. Is beats analytical or simulation-based. If simulation-based, then CTM is simulation-based? Most people would disagree, because the line is very fuzzy, and really the distinction makes little sense (as our framework demonstrates). Instead lets focus on modularity: All others (I hope) bake in the model, cost function, and solver.}

\ggnote{The list needs to be fleshed out a bit. For each, can we answer these questions: what is the problem being solved? What is the model? What is the solver algorithm? Then we can categorize them in some way.}

\begin{itemize}
\item DTALite \cite{zhou2014dtalite} is a DTA software package with a queue-based traffic simulator. DTALite uses agent-based algorithm to calculate dynamic traffic equilibrium.
\item Dynameq \cite{mahut2010traffic} software provides a vehicle-based traffic simulation with capacity to determine dynamic user equilibrium.  
\item DynaMIT-P \cite{DynaMIT,ben2001dynamit} is simulation-based DTA system designed to evaluate transportation
systems at the planning level. It comprises a demand and a supply simulator that communicate to produce a user equilibrium routing guidance using the rolling horizon process. 
\item DYNASMART-P \cite{DYNASMART,mahmassani2004dynasmart} is a simulation-based DTA tool based on the the mesoscopic traffic simulator DYNASMART. 
\item DynusT \cite{chiu2011dynust} is a simulation-based DTA software that use a gap function vehicle-based solution algorithm to determine user equilibrium.
\item INTEGRATION \cite{rakha2012integration} is DTA tool that uses a trip-based microscopic traffic model. It includes several methods to calculate DTA user equilibrium. 

\end{itemize}

In contrast to these, the software described here does not prescribe the model, cost-function, or numerical method, but rather serves as a test platform for comparing different options, both in terms of computation and the solutions they produce. The code can be found here \cite{ta_solver}.


%%%%%%%%%%%%%%%%%%%%%%%%%%%%%%%%%%%%%%%%%%%%%%%%%%%

% In addition, VI's extension and sensitivity analysis can be conveniently performed \cite{peeta2001foundations}. 
% [GG: This sounds like it might be important, but I am not sure what it is, or if it compares favorably to other sensitivity analyses, eg lagrange multipliers in optimization problems]

% Though the theory of DTA is still relatively undeveloped, variational inequality (VI) has proven to provide a unified formulation to address various classes of equilibrium and equivalent optimization problems in the DTA context. 
% [GG: I incorporated this idea in the text]
 
%Static methods were developed for long term transportation planning and are not applicable in dynamic route guidance systems that require the ability to solve transportation problems in real time\cite{boyce1989route}. 
% [GG: This is too strong of an assertion for my taste]

% In recent years, dynamic traffic assignment (DTA) has gained heightened interest, as researchers and practitioners are recognizing the need to predict the spatio-temporal evolution of traffic and the limitations of  static traffic assignment methods\cite{peeta2001foundations}. 
% [GG: The reference here is too old to be recent. Is there something more recent?]

% In this paper, we extend VI formulation to unify analytical DTA models with simulation-based DTA models. 
% [GG: I dont want to say that we did it, because most people wouldnt consider CTM a simulation. I'd rather say that we created a framework that will support it.]

% Simulation-based DTA methods compliment analytical techniques by addressing the shortcomings of analytical link models and exit functions in reproducing dynamic traffic interactions, satisfying First-In-Fist-Out property, and replicating complex vehicles and multi-user class interactions. 
% [GG: The line between "analytical" and "simulation-based" is artificial. I think we should move beyond and speak at a higher level than that false debate. I dont understand the difference between "analytical" and "simulation-based" anyway.]

% VI formulations can model more realistic traffic scenario by bypassing the issues of solution intractability associated with other analytical DTA methods such as mathematical programming and optimal control formulations. 
% [GG: This is not true.]


% our software is the first to combine simulation-based DTA models with analytical methods including mathematical programming and optimal control into one software platform. In addition, our software framework is based on a clear underlying VI DTA formulation. The general VI formulation determines the overall software architecture, but also provide a way to incorporate any type of DTA problem that can be represented with VI. 
% [GG : I really would rather not make grand claims. The clarity we've gained is only relative to my ignorance when we started. But there are people for whom everything we say here is obvious.]