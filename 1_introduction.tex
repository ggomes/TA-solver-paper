The term \textit{traffic assignment} captures an array of problems concerning the distribution of traffic over a network. Traffic assignment problems arise in transportation planning applications, such as in the traditional ``four-step'' procedure, where it occupies the final step, following trip generation, trip distribution, and mode choice \cite{mcnally2007four}.  The fundamental principle that guides most traffic assignment models is Wardrop's first principle \cite{wardrop1952some}, which states that among available alternatives, drivers will select routes that minimize their travel times. Because the speed of traffic tends to decrease as the number of vehicles on the route increases, this principle leads to an iterated game in which drivers test new routes, day after day, until they converge to a Nash equilibrium, a.k.a. a \textit{user equilibrium}. 

Since the pioneering work of Wardrop, the mathematical and numerical study of traffic assignment has progressed slowly.
This is perhaps due to a historical scarcity of traffic data and network information, but also because of the complexity of traffic behavior. \cite{peeta} gives several examples in which problems could not be solved for networks of useful size. It is not surprising therefore that the problem has received renewed interest with recent increases in traffic and network data, and computational power. Also driving this renewal are concerns about the `unintended consequences' of the widespread use of routing apps \cite{traffic_apps}, as well as advances in autonomous driving technologies.

An investigation into the techniques of traffic assignment reveals a multitude of traffic flow models, numerical methods, and performance metrics. As is usually the case, the more detailed the model, the more expensive the computation. However it is not clear under which circumstances the additional effort (or investment in computational resources) is necessary. For example, although dynamic models that capture essential congestion phenomena are desirable, the computation (as well as network configuration effort) involved in solving dynamic traffic assignment (DTA) problems can be large. On the other hand, existing static planning methods cannot fully model traffic congestion because of their inability to represent queue formation and dissipation when traffic demand exceeds road capacities\cite{nie2010solving}. 

This paper presents a software framework for addressing such questions. The software is based on the formulation of traffic assignment as a variational inequality, described by Nagurney in \cite{nagurney2013network}. This is a very general formulation, as it requires only continuity of the traffic model. Hence, it covers a wide range of models: static and dynamic, macroscopic and microscopic, `analytical' and `simulation-based', etc. The software is modular and extensible, in the sense that it provides interfaces for incorporating new models, cost functions, and numerical solvers. 

There are many other programs, both commercial and shared, that solve traffic assignment equilibrium problems. Our software is most related to software tools for dynamic traffic assignment problems. Programs such as 
DTALite \cite{zhou2014dtalite}, 
Dynameq \cite{mahut2010traffic}, 
DynaMIT \cite{DynaMIT, ben2001dynamit}, 
% DynaSMART \cite{mahmassani2004dynasmart}, 
% DynusT \cite{chiu2011dynust}, 
% Integration \cite{rakha2012integration}, 
etc. offer a variety of approaches to solving dynamic user equilibrium problems. In contrast to these, the software described here does not prescribe the model, cost-function, or numerical method, but rather serves as a platform for comparing different options in terms of computation and the solutions they produce. The software presently includes three traffic models, three travel time functions, and three solvers.
%static model, two dynamic models: the cell-transmission model and the Merchant-Nemhauser model, a travel time based cost-function, and three numerical algorithms: the Frank-Wolfe algorithm, the Method of Successive Averages, and the Extra-Projection Method. 
The code can be found here \cite{ta_solver}. 

